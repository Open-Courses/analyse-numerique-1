%        File: Anum.MA1.Cours.tex
%     Created: Mit Sep 30 05:00  2015 C
% Last Change: Mit Sep 30 05:00  2015 C
%
\documentclass[a4paper, 11pt]{report}

\usepackage[french]{babel}
\usepackage[T1]{fontenc}
\usepackage[utf8]{inputenc}
\usepackage{hyperref}

\usepackage{amsmath}
\usepackage{amsthm}
\usepackage{amssymb}
\usepackage{physics}

\usepackage{amsmath}

%-------------------------------------------------------------------------------
\newcommand{\naturel}{\mathbb{N}}
\newcommand{\integer}{\mathbb{Z}}
\newcommand{\rational}{\mathbb{Q}}
\renewcommand{\real}{\mathbb{R}} %Already defined in physics package
\newcommand{\complex}{\mathbb{C}}
%-------------------------------------------------------------------------------

%Logical
\def\implies{\Rightarrow}
\def\equiv{\Leftrightarrow}
%-------------------------------------------------------------------------------

% Calculability
\def\recursiveFunctionsSet{\mathcal{R}}
\def\primitiveRecursiveFunctionSet{\mathcal{RP}}
\def\muRecursiveFunctionSet{\mathcal{R}_{\mu}}
%-------------------------------------------------------------------------------

%Set theory
\def\union{\cup} %Union
\def\inter{\cap} %Intersection
\newcommand{\comp}[1]{#1^{c}} %Complementary
\def\cartprod{\cross}
\newcommand{\cardinal}[1]{|#1|}
%-------------------------------------------------------------------------------

%Topology
\def\interior{\mathring}
\def\adh{\overline}
%-------------------------------------------------------------------------------

%Algebra

% Linear algebra

\newcommand{\matrixSpace}[2]{M_{#1}(#2)}
\newcommand{\inversibleMatrixSpace}[2]{GL_{#1}(#2)}
\newcommand{\spanspace}[1]{\left<#1\right>}

% Group theory

\def\isomorphe{\simeq}
\newcommand{\ordergroup}[1]{|#1|}
\newcommand{\GSautomorphismDef}[2]{Aut_{\text{#1}}(#2)}
\newcommand{\generatedGroup}[1]{\left<#1\right>}
\newcommand{\permutationGroup}[1]{\mathfrak{S}(#1)}

% Field theory

%extensionDegree
\newcommand{\extensionDegree}[2]{[#1 : #2]}

\newcommand{\extensionField}[2]{#1/#2}

%plongement
\newcommand{\plongement}[3]{Hom_{#1}(#2, #3)}

%Galois group
\newcommand{\galoisGroup}[2]{G(#1, #2)}

% Spectrum Theory
\newcommand{\spectrum}[1]{\sigma(#1)}
\newcommand{\resolvant}[1]{\rho(#1)}

\def\Ldeux{\mathcal{L}^{2}}
\def\Ldeuxstar{(\mathcal{L}^{2})^{*}}

%GSsequence :
%		#1 : represention of elements of the sequences
%		#2 : indices
%		#3 : set definition
\newcommand{\GSsequence}[3]{(#1_{#2})_{#2 \in #3}}

%GSsetDef :
%		#1 : set elements
\newcommand{\GSset}[1]{\left\{ #1 \right\}}

%GSsetDef :
%		#1 : global set
%		#2 : condition
\newcommand{\GSsetDef}[2]{\left\{#1 \, | \, #2 \right\}}

%GSprodSet :
%		#1 : indice
%		#2 : begin indice
%		#3 : end indice
%		#4 : set
\newcommand{\GSprodSet}[4]{\displaystyle \prod_{#1 = #2}^{#3} #4_{#1}}

%GSsum :
%		#1 : indice
%		#2 : begin indice
%		#3 : end indice
%		#4 : element
\newcommand{\GSsum}[4]{\displaystyle \sum_{#1 = #2}^{#3} #4}

\newcommand{\GSintervalCC}[2]{\left[#1, #2\right]}
%-------------------------------------------------------------------------------

%Analysis :

% conjuguate
\def\conjuguate{\overline}

% miLength: multi-indice length
\newcommand{\miLength}[1]{|#1|}

% segment: all points between two points given
\newcommand{\segment}[2]{S(#1, #2)}

%GSfunction :
%       #1 : name function
%       #2 : begin set
%       #3 : end set
\newcommand{\GSfunction}[3]{#1 : #2 \rightarrow #3}

%GSnorme : Deprecated --> \norm
%		#1 : elements which norme is applied on
\newcommand{\GSnorme}[1]{\norm{#1}}

%GSnormeDef :
%		#1 : elements which norme is applied on
%		#2 : norme indice
\newcommand{\GSnormeDef}[2]{\norm{#1}_{#2}}

%GSnormedSpace :
%		#1 : vectorial space
%		#2 : \GSnorme[Def] with dot as element.
\newcommand{\GSnormedSpace}[2]{(#1, #2)}

%Identification
\def\identification{\simeq}

%GSdual
%		#1 : vectorial space
\newcommand{\GSdual}[1]{#1^{*}}

%GSbidual
%		#1 : vectorial space
\newcommand{\GSbidual}[1]{#1^{**}}

\newcommand{\GSunitBoule}[1]{\mathcal{B}_{#1}}
\newcommand{\GSclosedUnitBoule}[1]{\adh{\GSunitBoule{#1}}}

\newcommand{\GSweakTopo}[1]{\sigma(#1, #1^{*})}
\newcommand{\GSpreweakTopo}[1]{\sigma(#1^{*}, #1)}

%GSendomorphism
\newcommand{\GSendomorphism}[1]{End(#1)}

%GShomomorphisme
\newcommand{\GShomomorphisme}[3][]
{
	Hom_{#1}(#2, #3)
}

%GShomomorphismeDef
% Deprecated !! Use instead directly \GShomomorsphisme
\newcommand{\GShomomorphismeDef}[3][]
{
	\GShomomorphisme{#1}{#2}{#3}
}

%GScontinueEndo
\newcommand{\GScontinueEndo}[2][]
{
	\mathcal{L}_{#1}(#2)
}

%\GScontinueHomo
\newcommand{\GScontinueHomo}[3][]
{
	\mathcal{L}_{#1}(#2; #3)
}

%\GScompactEndo
\newcommand{\GScompactEndo}[1]{\mathcal{K}(#1)}

%\GScompactHomo
\newcommand{\GScompactHomo}[2]{\mathcal{K}(#1; #2)}

\newcommand{\GSfiniteRankHomo}[2]{\mathcal{R}_{f}(#1; #2)}
\newcommand{\GSfiniteRankEndo}[1]{\mathcal{R}_{f}(#1)}

\newcommand{\GSisomorphisme}[1]{Isom(#1)}
\newcommand{\GSisomorphismeHomo}[2]{Isom(#1; #2)}
\newcommand{\GSisometryEndo}[1]{Isom(#1)}

\newcommand{\jacobienneMatrix}[2]{J_{#1}(#2)}
\newcommand{\hessienneMatrix}[2]{\mathcal{H}_{#1}(#2)}
%-------------------------------------------------------------------------------
%Model theory

\def\satisfies{\vdash}
\newcommand{\lang}[1]{\mathcal{#1}}
\newcommand{\theory}[1]{\mathcal{#1}}
\newcommand{\struct}[1]{\mathcal{#1}}

%Definissable set
%	1 : order of the definissable sets
%	2 : the l-structure which we define the definissable sets on.
\newcommand{\definissableSet}[2][]
{
	Def^{#1}(#2)
}

%Type set
%	1 : n if it's the set of n-types.
%	2 : the theory which we build the types on.
\newcommand{\typeSet}[2]{S_{#1}(#2)}

%Ultraproduct
%	1 : indice elements
%	2 : set which contains indices
%	3 : ultrafilter
%	4 : models represention
\newcommand{\GSultraproduct}[4]{\displaystyle {\prod_{#1 \in #2}}^{#3}#4_{#1}}

%Ultrapower
%	1 : indice elements
%	2 : set which contains indices
%	3 : ultrafilter
%	4 : model
\newcommand{\ultrapower}[4]{\displaystyle {\prod_{#1 \in #2}}^{#3} #4}

%Substructures
\def\substructure{\subseteq}

%Elementary Substructures.
\def\elemSubstructure{\preceq}

\def\existentiallyClosed{\underset{e.c}{\subseteq}}
%-------------------------------------------------------------------------------

%	Hilbert space
\def\Hilbert{\mathcal{H}}
\newcommand{\GSortho}[1]{#1^{\perp}}
\def\GSid{\cong}
\newcommand{\dotprod}[2]{\bra{#1}\ket{#2}}
\newcommand{\adjointe}[1]{#1^{*}}
%-------------------------------------------------------------------------------

%	Group representions
\newcommand{\GSrepr}[2]{Repr(#1, #2)}
\newcommand{\GSreprf}[2]{Repr_{f}(#1, #2)}
\newcommand{\GSrepri}[2]{Repr_{i}(#1, #2)}
%-------------------------------------------------------------------------------

%Probability

%	Borelian
\newcommand{\borelian}[1]{\mathcal{B}(#1)}

%	Laws
\newcommand{\lawBernouilli}[1]{\mathcal{B}(1, #1)}
\newcommand{\lawBinomial}[2]{\mathcal{B}(#1, #2)}
\newcommand{\lawExponential}[1]{e(#1)}
\newcommand{\lawPoisson}[1]{\mathcal{P}(#1)}
\newcommand{\lawNormal}[2]{\mathcal{N}(#1, #2)}
\newcommand{\lawUniform}[1]{\mathcal{U}(#1)}
\newcommand{\lawCauchy}{\mathcal{C}}

%	Variables
\def\varFollow{\sim}
\def\varSameLaw{\overset{\mathcal{L}}{=}}
\def\varIndependant{\perp}

%-------------------------------------------------------------------------------

\usepackage{amsfonts}
\usepackage{amssymb}
\usepackage{amsmath}
\usepackage{amsthm}
\usepackage{mathrsfs}

\newtheorem{definition}{Définition}[chapter]

\newtheorem{proposition}[definition]{Proposition}
\newtheorem{lemma}[definition]{Lemme}
\newtheorem{corollary}[definition]{Corollaire}
\newtheorem{theorem}[definition]{Théorème}

\newtheorem*{exemple}{Exemple}
\newtheorem*{question}{Questions}
\newtheorem*{remarque}{Remarque}
\newtheorem*{notation}{Notation}

\newtheorem{exercice}{Exercice}[chapter]


\title{Analyse numérique}
\author{Danny Willems}

\begin{document}

\maketitle

\tableofcontents

\chapter{Equations différentielles ordinaires}

\section{Définitions}
	\begin{definition} [Equation différentielle ordinaire]
		Soit $\Omega$ un ouvert de $\real \cartprod E$.

		Soit $f : \Omega \rightarrow E$ une application.

		On appelle \textbf{équation différentielle ordinaire d'ordre 1} une équation
		du type
		\begin{equation}
			\frac{dx}{dt} = f(t, x)
		\end{equation}
		noté plus souvent
		\begin{equation}
			\dot{x} = f(t, x)
		\end{equation}
	\end{definition}

   % On remarque que si on explicite $\vec{x}$ en $(x_{1}, \dots, x_{n})$ et $f(t,
	%\vec{x})$ en $(f_{1}(t, \vec{x}), \dots, f_{n}(t, \vec{x}))$, on peut écrire, en
	%identifiant composante par composante, une
	%équation différentielle d'ordre 1 de la manière suivante
	%\begin{align}
		%\dot{x_{1}} & = f_{1}(t, \vec{x}) \\
		%& \dots \\
		%\dot{x_{n}} & = f_{n}(t, \vec{x})
	%\end{align}

	%On dit que ce système est \textbf{sous forme normale} car il se résout par
	%rapport aux dérivées. % ???

	%On obtient alors un \textbf{système d'équations différentielles ordinaires
	%d'ordre 1}. Résoudre ce système revient alors à résoudre l'équation
	%différentielle ordinaire d'ordre 1 de départ.

	\begin{definition} [Solution d'une équation différentielle ordinaire d'ordre
		$1$]
		Soit $\Omega$ un ouvert de $\real \cartprod E$.

		Soit $f : \Omega \rightarrow E$ une application et
		\begin{equation}
			\label{eq:def_sol_eq_diff}
			\frac{dx}{dt} = f(t, x)
		\end{equation}
		une équation différentielle ordinaire d'ordre 1.

		On appelle \textbf{solution de l'équation \ref{eq:def_sol_eq_diff}} toute
		fonction
		\begin{equation}
			x : I \rightarrow E
		\end{equation}
		où $I$ est un intervalle non vide de $\real$
		tel que pour tout $t \in I$,
		\begin{equation}
			(t, x(t)) \in \Omega
		\end{equation}
		et
		\begin{equation}
			\dot{x}(t) = f(t, x(t))
		\end{equation}
		où
		\begin{equation}
			\dot{x}(t) = \dv{x}{t}
		\end{equation}

		L'ensemble
		\begin{equation}
			\GSsetDef{(t, x(t))}{t \in I}
		\end{equation}
		est appelé \textbf{trajectoire de la solution $x$ de
		l'équation \ref{eq:def_sol_eq_diff}} ou \textbf{espace de mouvement}.

		L'ensemble
		\begin{equation}
			\GSsetDef{x(t)}{t \in I}
		\end{equation}
		est appelé \textbf{orbite de la solution $x$ de
		l'équation \ref{eq:def_sol_eq_diff}} ou \textbf{espace de phase}.
	\end{definition}

	Remarquons que l'intervalle d'une solution peut être quelconque
	topologiquement parlant. Il peut être ouvert, fermé, ou aucun des deux.

	\begin{definition} [Problème de Cauchy]
		Soit $\Omega$ un ouvert de $\real \cartprod E$.

		Soient $f : \Omega \rightarrow E$ une application,
		\begin{equation}
			\label{eq:def_cauchy_sol_diff}
			\frac{dx}{dt} = f(t, x)
		\end{equation}
		une équation différentielle ordinaire d'ordre 1 et $(t_{0},
		x_{0})$ un point de $\Omega$.

		On appelle \textbf{problème de Cauchy de l'équation
		\ref{eq:def_cauchy_sol_diff} relativement aux conditions initiales
		$(t_{0}, x_{0})$} la recherche des solutions
		\begin{equation}
			x : I \rightarrow E
		\end{equation}
		de l'équation \ref{eq:def_cauchy_sol_diff} tel que
		\begin{enumerate}
			\item $t_{0} \in I$
			\item $\dot{x}(t_{0}) = f(t_{0}, x_{0})$
		\end{enumerate}
	\end{definition}

	% Interprétation géométrique

	Nous allons donner la définition d'une équation différentielle ordinaire
	d'orde $n$. La définition d'équation différentielle d'ordre $1$ que nous
	avons donnée actuellement ne fait intervenir que la dérivée première d'une
	fonction $x$ et une dépendance par rapport à cette même fonction
	$x$ dans la fonction $f$.

	La dérivée de la fonction $x$ est alors donnée par la fonction
	$f$.

	Il est naturel de généraliser cette définition pour la $n$-ième dérivée.

	\begin{definition} [Equation différentielle ordinaire d'ordre $n$]
		Soit $n \in \naturel^{> 0}$ et soit $\Omega$ un ouvert de $\real
		\cartprod E$.

		Soit $f : \Omega \rightarrow \real$.

		On appelle \textbf{équation différentielle ordinaire d'ordre $n$} ou
		\textbf{équation différentielle ordinaire à dérivée $n$-ième explicitée}
		une équation du type
		\begin{equation}
			y^{(n)} = f(t, y^{(n - 1)}, y^{(n - 2)}, \dots, y^{(1)}, y^{(0)})
		\end{equation}

		où
		\begin{equation}
			y^{(0)} = y
		\end{equation}
		et
		\begin{equation}
			y^{(k)} = \dv[k]{y}{t}
		\end{equation}
	\end{definition}

	Remarquons que la définition d'une équation différentielle ordinaire
	d'ordre $n$ est définie par une fonction $f$ qui va de $\Omega$ dans
	$\real$, et non dans $E$ comme dans le cas d'une équation
	différentielle ordinaire d'ordre $1$.

	De la même manière que l'on a défini une solution d'une équation
	différentielle ordinaire d'ordre $1$, on définit la solution d'une équation
	différentielle ordinaire d'ordre $n$ de la façon suivante:

	\begin{definition} [Solution d'une équation différentielle ordinaire d'ordre
		$n$]
		Soit $n \in \naturel^{> 0}$ et soit $\Omega$ un ouvert de $\real
		\cartprod \real^{n}$.

		Soit $f : \Omega \rightarrow \real^{n}$ une application et
		\begin{equation}
			y^{(n)} = f(t, y^{(n - 1)}, y^{(n - 2)}, \dots, y^{(1)}, y^{(0)})
			\label{eq:def_eq_diff_n}
		\end{equation}
		une équation différentielle ordinaire d'ordre $n$.

		On appelle \textbf{solution de l'équation \ref{eq:def_eq_diff_n}} toute
		fonction
		\begin{equation}
			y : I \rightarrow \real
		\end{equation}
		$n$ fois dérivable où $I$ est un intervalle non vide de $\real$ tel que
		pour tout $t \in I$,
		\begin{enumerate}
			\item $(t, y^{(n - 1)}(t), y^{(n - 2)}(t), \dots, y^{(1)}(t),
				y^{(0)}(t)) \in \Omega$
				\label{eq:def_sol_eq_diff_set}
			\item $y^{(n)}(t) = f(t, y^{(n - 1)}(t), y^{(n - 2)}(t), \dots,
				y^{(1)}(t), y^{(0)}(t))$
				\label{eq:def_sol_eq_diff_egality}
		\end{enumerate}
	\end{definition}

	Remarquons que l'intervalle d'une solution peut être quelconque
	topologiquement parlant. Il peut être ouvert, fermé, ou aucun des deux. De
	plus, cette fois-ci, la définition d'une solution est une fonction allant de
	$I$ dans $\real$, non plus dans $E$

	L'étude des équations différentielles ordinaire d'ordre $n$ peut sembler
	difficile à première vue car nous devons vérifier que les dérivées
	successives d'une solution $y$ vérifient bien l'égalité
	\ref{eq:def_sol_eq_diff_egality}.

	Nous allons montrer une première proposition qui nous montre que l'étude des
	solutions d'une équation différentielle ordinaire d'ordre $n$ peut revenir à
	l'étude d'un système d'équations différentielles ordinaires d'ordre $1$.

	\begin{proposition}
		Soient $\Omega$ un ouvert de $\real \cartprod \real^{n}$, $f :
		\Omega \rightarrow \real$ et
		\begin{equation}
			y^{(n)} = f(t, y^{(n - 1)}, y^{(n - 2)}, \dots, y^{(1)}, y^{(0)})
			\label{eq:def_eq_diff_n_prop_simplify}
		\end{equation}
		une équation différentielle ordinaire d'ordre $n$.
		Alors il existe un système d'équation différentielle d'ordre $1$ dont
		l'ensemble de solutions est de même cardinal que l'ensemble de solution
		de l'équation \ref{eq:def_eq_diff_n_prop_simplify}.
	\end{proposition}

	\ifdefined\outputproof
	\begin{proof}

	\end{proof}
	\fi

\section{Fonctions lipschitziennes}

	\begin{definition} [Lipschitzienne]
		Soit $\Omega$ un ouvert de $\real \cartprod E$.

		Soit $f : \Omega \rightarrow E$ une application.
		On dit que $f$ est \textbf{lipschitzienne par rapport à la deuxième
		variable} s'il existe $k \geq 0$ tel que
		\begin{equation}
			\forall (t, x_{1}) \in \Omega, \, \forall (t, x_{2}) \in \Omega, \,
			\norm{f(t, x_{1}) - f(t, x_{2})} \leq k \norm{x_{1} - x_{2}}
		\end{equation}
	\end{definition}

	\begin{definition}
		Soit $\Omega$ un ouvert de $\real \cartprod E$.

		Soit $f : \Omega \rightarrow E$ une application.

		On dit que $f$ est \textbf{localement lipschitzienne par rapport à la
		deuxième variable} s'il existe $k \geq 0$ tel qu'en tout point $(t, x)$ de $\Omega$, il existe un
		voisinage de $(t, x)$ sur lequel $f$ est $k$-lipschitzienne.
	\end{definition}

	Montrons quelques propriétés que les fonctions lipschitziennes et localement
	lipschitziennes entretiennent avec les fonctions continues.

	\begin{proposition}
		Soit $\Omega$ un ouvert de $\real \cartprod E$ et soit $f :
		\Omega \rightarrow E$ une application.

		Si $f$ est lipschitzienne par rapport à la deuxième variable, alors $f$
		est uniformément continue par rapport à la deuxième variable.
	\end{proposition}

	\ifdefined\outputproof
	\begin{proof}

	\end{proof}
	\fi

	\begin{remarque}
		Si une fonction $f : \Omega \rightarrow E$ est lipschitzienne
		par rapport à la deuxième variable, cela n'implique pas nécéssairement
		qu'elle soit continue par rapport à la première variable et, par
		conséquence, uniformément continue par rapport à la première variable.
		% TODO: chercher exemple
	\end{remarque}

	\begin{proposition}
		Soit $\Omega$ un ouvert de $\real \cartprod E$ et $f : \Omega
		\rightarrow E$ une application tel que les dérivées partielles
		$\pdv{f}{x}$ par rapport à $x$ existe et sont continues.

		Alors, $f$ est localement lipschitzienne dans $\Omega$.
	\end{proposition}

	\ifdefined\outputproof
	\begin{proof}
		Soit $(t_{0}, x_{0}) \in \Omega$. Comme $\Omega$ est ouvert, il
		existe une boule fermée $B[ (t_{0}, x_{0}), r]$ de rayon $r$ et de
		centre $(t_{0}, x_{0})$ contenue dans $\Omega$.

		Par hypothèse, les dérivées premières $\pdv{f}{x}$ sont continues.
		Donc $\pdv{f}{x}$ sont bornées sur cette boule $B[ (t_{0},
		x_{0}), r]$.
	\end{proof}
	\fi

	\begin{proposition}
		Soit $\Omega$ un ouvert convexe de $\real \cartprod E$ et soit
		\begin{equation}
			f : \Omega \rightarrow E : (t, x) \rightarrow f(t,
			x)
		\end{equation}
		une application tel que les dérivées
		partielles $\pdv{f}{x}$ sont continues par rapport à $x$.
		LASSE.
		\begin{enumerate}
			\item $f$ est lipschitzienne.
			\item Les dérivées partielles $\pdv{f}{x}$ sont bornées.
		\end{enumerate}
	\end{proposition}

	\ifdefined\outputproof
	\begin{proof}

	\end{proof}
	\fi

\section{Tonneaux de sécurité}

\begin{definition}
	Un tonneau de centre $(t_{0}, x_{0})$ est un ensemble $I \cartprod B
	\subseteq \real \cartprod E$ où
	\begin{equation}
		I = \GSsetDef{t \in \real}{\abs{t - t_{0}} < l}
	\end{equation}
	et
	\begin{equation}
		B = \GSsetDef{x \in E}{\GSnorme{x - x_{0}} < M}
	\end{equation}

	La valeur $2l$ est appelée \textbf{longueur du tonneau} et $r$ \textbf{le
	rayon du tonneau}.
\end{definition}

\begin{definition}
	Soit $\Omega$ un ouvert de $\real \cartprod E$ et soit $f : \Omega
	\rightarrow E$ une application.

	Soient $(t_{0}, x_{0})$ un point de $\Omega$ et $T$ un tonneau de
	centre $(t_{0}, x_{0})$.

	$T$ est \textbf{un tonneau de sécurité de centre $(t_{0}, x_{0})$
relativement à $f$} si $T$ est contenu dans $\Omega$ et si $r = Ml$ où $M$ est
une borne supérieure de $\GSnormeDef{f(t, x)}{E}$ où $(t, x)$ parcourt $T$.

	$T$ est \textbf{un tonneau lipschitzien relativement à $f$} si $f$ est
	lipschitzienne sur $T$.
\end{definition}

\begin{proposition}
	Soit $\Omega$ un ouvert de $\real \cartprod E$ et soit $f : \Omega
	\rightarrow E$ une application continue et localement lipschitizienne en la
	deuxième variable.

	Soient $(t_{0}, x_{0})$ un point de $\Omega$ et $T$ un tonneau de
	centre $(t_{0}, x_{0})$ contenue dans $\Omega$.

	Alors $T$ est contenu dans un tonneau de sécurité de centre $(t_{0},
	x_{0})$.
\end{proposition}

\ifdefined\outputproof
\begin{proof}

\end{proof}
\fi

\begin{proposition}
	Soit $\Omega$ un ouvert de $\real \cartprod E$ et soit $f : \Omega
	\rightarrow E$ une application continue et localement lipschitzienne.

	Soit $(t_{0}, x_{0})$ un point de $\Omega$.

	Alors il existe un tonneau de sécurité lipschitzien de centre $(t_{0},
	x_{0})$ relativement à $f$.
\end{proposition}

\ifdefined\outputproof
\begin{proof}

\end{proof}
\fi

\section{Existence et unicité locales de solutions}

Posons $\Omega$ un ouvert de $\real \cartprod E$ et $f : \Omega
\rightarrow E$ une application localement lipschitzienne en
$x$ et continue.

Soit
\begin{equation}
	\dot{x} = f(t, x)
	\label{eq:existence_sol}
\end{equation}
l'équation différentielle ordinaire d'ordre un régie par $f$.

Nous allons montrer le théorème suivant:

\begin{theorem}
	Soit $(t_{0}, x_{0}) \in \Omega$.

	Alors il existe un intervalle $I \subseteq \real$ et une solution $x :
	I \rightarrow E$ de \ref{eq:existence_sol} tel que $x(t_{0}) =
	x_{0}$.

	De plus, pour tout autre intervalle $J \subseteq I \subseteq \real$ avec
	$t_{0} \in J$ et pour toute autre solution $y : J \rightarrow
	E$ tel que $y(t_{0}) = x_{0}$, on a
	\begin{equation}
		\forall t \in J, x(t) = y(t)
	\end{equation}
\end{theorem}

Cependant, avant de montrer ce dernier, nous allons démontrer des propositions
qui nous permettront de conclure sur ce théorème.

\begin{proposition}
	Soit $(t_{0}, x_{0}) \in \Omega$ et soit $I$ un intervalle de $\real$ tel que $t_{0}
	\in I$.

	Soit $x : I \rightarrow E$ une fonction tel que son graphe est $\Omega$.

	Les assertions suivantes sont équivalentes:
	\begin{enumerate}
		\item $x$ est solution de l'équation \ref{eq:existence_sol} et $x(t_{0})
			= x_{0}$
		\item $x$ est continue et pour tout $t \in I$,
			\begin{equation}
				x(t) = x_{0} + \int_{t_{0}}^{t} f(\tau, x(\tau)) d\tau
			\end{equation}
	\end{enumerate}
\end{proposition}

\ifdefined\outputproof
\begin{proof}

\end{proof}
\fi

\begin{proposition}
	Soient $I$ un intervalle de $\real$ et $T = I \cartprod B$ un tonneau de
	sécurité de lipschitzien de $f$.

	Soit $C(I)$ l'ensemble des fonctions $x
	: I \rightarrow E$ continue et dont le graphe est dans $T$.

	Alors $C(I)$ est un espace métrique complet pour la métrique de la
	convergence uniforme.
\end{proposition}

\ifdefined\outputproof
\begin{proof}

\end{proof}
\fi

\begin{proposition}
	Soit $C(I)$ l'ensemble défini précédemment.

	Alors la fonction
	\begin{equation}
		\phi : C(I) \rightarrow C(I)
	\end{equation}
	tel que pour tout $x \in C(I)$ et pour tout $t \in I$,
	\begin{equation}
		\phi(x)(t) = x_{0} + \int_{t_{0}}^{t} f(\tau, x(\tau)) d\tau
	\end{equation}
	est bien définie et est contractante.
\end{proposition}

\ifdefined\outputproof
\begin{proof}

\end{proof}
\fi

\begin{corollary}
	Il existe une unique fonction $y \in C(I)$ tel que
	\begin{equation}
		\forall t \in I, \, y(t) = x_{0} + \int_{t_{0}}^{t} f(\tau, y(\tau))
		d\tau
	\end{equation}
\end{corollary}

\ifdefined\outputproof
\begin{proof}

\end{proof}
\fi

On obtient alors le théorème d'existence locale d'une solution

\ifdefined\outputproof
\begin{proof} [Théorème d'existence locale d'une solution]
	On a montré qu'il existe une unique fonction $y : I \rightarrow E$ tel que
	$y$ est contenue, dont le graphe est contenue dans $T$ et

	\begin{equation}
		\forall t \in I, \, y(t) = x_{0} + \int_{t_{0}}^{t} f(\tau, y(\tau))
		d\tau
	\end{equation}

	Ce qui est équivalent à dire que $y$ est solution de l'équation
	\ref{eq:existence_sol}.
\end{proof}
\fi

\section{Existence et unicité globale d'une solution}

Nous allons montrer que toute solution se prolonge en une solution maximale et
que dans les hypothèses de localité lipschitzienne et de continuité, cette
solution est maximale.

Donnons d'abord la définition de solution maximale.


\end{document}
